\documentclass{dcbl/challenge}

\setdoctitle{Working with getopt}
\setdocauthor{Stephan Bökelmann}
\setdocemail{sboekelmann@ep1.rub.de}
\setdocinstitute{AG Physik der Hadronen und Kerne}
\usepackage{listings}

\begin{document}

When a program is loaded into RAM, the whole string, that called the program is stored in the new processes RAM.
The C-Compiler knows where they are located and can therefore call them.
These strings are called \textbf{command line arguments}.
Take a close look at a full definition of a main-function:
\begin{lstlisting}
int main(int argc, char **argv) {}
\end{lstlisting}
The shell that calls the new program copies the following information into the new process:
\begin{enumerate}
    \item The number of words, given to the commandline and separeted by spaces
    \item A pointer to a pointer to the words
\end{enumerate}
The second parameter looks a little bit weird, so let's investigate it further.
A character is stored in memory by using the type \texttt{char}.
So how would we deal with a bunch of consecutive characters? 
The answer to that is the string-mechanic. 
A string is nothing but an array of characters, which are stored contiguously in memory.
The array is terminated by a null character (\texttt{\\0}).
Since an array passed to a function is usually passed as a pointer to the first element, plus the number of elements in that array, we can now understand the idea of the parameters to main.
\texttt{char **argv} is a pointer to the first string. 
Thus the second parameter is \texttt{argv[1]}, and so on and so forth.
\section*{Exercise}
\begin{aufgabe}
    Write a program, that prints the number of arguments.
\end{aufgabe}

\begin{aufgabe}
    Write a program, that takes multiple arguments, and prints them.
\end{aufgabe}

\begin{aufgabe}
    Now we want to use the standard library function getopt.
    This function is defined in the header file \texttt{unistd.h}.
    Check the following examples, clone and compile them: \url{https://github.com/bjoekeldude/uebergabe_an_main} 
    After that, write a program, that takes a flag.
    If the flag is set, print the number of arguments.
\end{aufgabe}

\begin{aufgabe}
    Write a program, that takes an option with an integer value.
    The program should also take multiple parameters. 
    The number given by the option, determines which parameter is printed.
    If the number is greater than the number of parameters, print all parameters.
\end{aufgabe}

\section*{Annotations}
\begin{enumerate}
    \item Example-Repo: \url{https://github.com/bjoekeldude/uebergabe_an_main}
\end{enumerate}

\end{document}
